% Special formatting for Computers and Typesetting: The Programs

\xdef\fmtversion{\fmtversion+TWIMAC}

\newif\ifsorted
\newread\sreffile \openin\sreffile=\jobname.sref
\newwrite\reffile
\ifeof\sreffile \immediate\openout\reffile=\jobname.ref \sortedfalse
  \message{This is the first pass!}
\else \sortedtrue \message{This is the second pass!} \fi

\newdimen\em \em=10pt
\parskip 0pt % no stretch between paragraphs
\parindent 1\em % for paragraphs and for the first line of Pascal text

\font\manual=manfnt
\font\ninerm=cmr9
\font\eightrm=cmr8
\font\sixrm=cmr6
\font\ninei=cmmi9
\font\eighti=cmmi8
\font\sixi=cmmi6
\skewchar\ninei='177 \skewchar\eighti='177 \skewchar\sixi='177
\font\ninesy=cmsy9
\font\eightsy=cmsy8
\font\sixsy=cmsy6
\skewchar\ninesy='60 \skewchar\eightsy='60 \skewchar\sixsy='60
\font\ninebf=cmbx9
\font\eightbf=cmbx8
\font\sixbf=cmbx6
\font\ninett=cmtt9
\font\eighttt=cmtt8
\hyphenchar\ninett=-1 \hyphenchar\eighttt=-1
\font\ninesl=cmsl9
\font\eightsl=cmsl8
\font\nineit=cmti9
\font\eightit=cmti8

\newif\iftenpoint
\def\tenpoint{\tenpointtrue
 \def\rm{\fam0\tenrm}%
  \textfont0=\tenrm \scriptfont0=\sevenrm \scriptscriptfont0=\fiverm
  \textfont1=\teni \scriptfont1=\seveni \scriptscriptfont1=\fivei
  \textfont2=\tensy \scriptfont2=\sevensy \scriptscriptfont2=\fivesy
  \textfont3=\tenex \scriptfont3=\tenex \scriptscriptfont3=\tenex
  \def\it{\fam\itfam\tenit}%
  \textfont\itfam=\tenit
  \def\sl{\fam\slfam\tensl}%
  \textfont\slfam=\tensl
  \def\bf{\fam\bffam\tenbf}%
  \textfont\bffam=\tenbf \scriptfont\bffam=\sevenbf
   \scriptscriptfont\bffam=\fivebf
  \def\tt{\fam\ttfam\tentt}%
  \textfont\ttfam=\tentt
  \def\ttx{\tentex}%
  \normalbaselineskip=12pt
  \def\MF{{\manual META}\-{\manual FONT}}%
  \let\sc=\eightrm
  \let\big=\tenbig
  \setbox\strutbox=\hbox{\vrule height8pt depth3pt width 0pt}%
  \normalbaselines\rm}

\def\ninepoint{\tenpointfalse
 \def\rm{\fam0\ninerm}%
  \textfont0=\ninerm \scriptfont0=\sixrm \scriptscriptfont0=\fiverm
  \textfont1=\ninei \scriptfont1=\sixi \scriptscriptfont1=\fivei
  \textfont2=\ninesy \scriptfont2=\sixsy \scriptscriptfont2=\fivesy
  \textfont3=\tenex \scriptfont3=\tenex \scriptscriptfont3=\tenex
  \def\it{\fam\itfam\nineit}%
  \textfont\itfam=\nineit
  \def\sl{\fam\slfam\ninesl}%
  \textfont\slfam=\ninesl
  \def\bf{\fam\bffam\ninebf}%
  \textfont\bffam=\ninebf \scriptfont\bffam=\sixbf
   \scriptscriptfont\bffam=\fivebf
  \def\tt{\fam\ttfam\ninett}%
  \textfont\ttfam=\ninett
  \def\ttx{\ninetex}%
  \normalbaselineskip=11pt
  \def\MF{{\manual hijk}\-{\manual lmnj}}%
  \let\sc=\sevenrm
  \let\big=\ninebig
  \setbox\strutbox=\hbox{\vrule height8pt depth3pt width 0pt}%
  \normalbaselines\rm}

\def\eightpoint{%
 \def\rm{\fam0\eightrm}%
  \textfont0=\eightrm \scriptfont0=\sixrm \scriptscriptfont0=\fiverm
  \textfont1=\eighti \scriptfont1=\sixi \scriptscriptfont1=\fivei
  \textfont2=\eightsy \scriptfont2=\sixsy \scriptscriptfont2=\fivesy
  \textfont3=\tenex \scriptfont3=\tenex \scriptscriptfont3=\tenex
  \def\it{\fam\itfam\eightit}%
  \textfont\itfam=\eightit
  \def\sl{\fam\slfam\eightsl}%
  \textfont\slfam=\eightsl
  \def\bf{\fam\bffam\eightbf}%
  \textfont\bffam=\eightbf \scriptfont\bffam=\sixbf
   \scriptscriptfont\bffam=\fivebf
  \def\tt{\fam\ttfam\eighttt}%
  \textfont\ttfam=\eighttt
  \def\ttx{\eighttex}%
  \normalbaselineskip=9pt
  \def\MF{{\manual opqr}\-{\manual stuq}}%
  \let\sc=\sixrm
  \let\big=\eightbig
  \setbox\strutbox=\hbox{\vrule height7pt depth2pt width 0pt}%
  \normalbaselines\rm}

\def\tenbig#1{{\hbox{$\left#1\vbox to8.5pt{}\right.\nulldelimiterspace=0pt$}}}
\def\ninebig#1{{\hbox{$\textfont0=\tenrm\textfont2=\tensy
  \left#1\vbox to7.25pt{}\right.\nulldelimiterspace=0pt$}}}
\def\eightbig#1{{\hbox{$\textfont0=\ninerm\textfont2=\ninesy
  \left#1\vbox to6.5pt{}\right.\nulldelimiterspace=0pt$}}}

\font\tentex=cmtex10
\font\ninetex=cmtex9 % TeX extended character set (used in strings)
\font\eighttex=cmtex8

\def\\#1{\hbox{\it#1\/\kern.05em}} % italic type for identifiers
\def\|#1{\hbox{$#1$}} % one-letter identifiers look a bit better this way
\def\&#1{\hbox{\bf#1\/}} % boldface type for reserved words
\def\.#1{\hbox{\ttx % typewriter type for strings
  \let\\=\BS % backslash in a string
  \let\'=\RQ % right quote in a string
  \let\`=\LQ % left quote in a string
  \let\{=\LB % left brace in a string
  \let\}=\RB % right brace in a string
  \let\~=\TL % tilde in a string
  \let\ =\SP % space in a string
  \let\_=\UL % underline in a string
  \let\&=\AM % ampersand in a string
  #1}}
\def\#{\hbox{\tt\char`\#}} % parameter sign
\def\${\hbox{\tt\char`\$}} % dollar sign
\def\%{\hbox{\tt\char`\%}} % percent sign
\def\^{\ifmmode\mathchar"222 \else\char`^ \fi} % pointer or hat
% circumflex accents can be obtained from \^^D instead of \^
\def\AT!{@} % at sign for control text
\def\pct!{{\char`\%}} % percent sign in ordinary text

\chardef\AM=`\& % ampersand character in a string
\chardef\BS=`\\ % backslash in a string
\chardef\LB=`\{ % left brace in a string
\def\LQ{{\tt\char'22}} % left quote in a string
\chardef\RB=`\} % right brace in a string
\def\RQ{{\tt\char'23}} % right quote in a string
\def\SP{{\tt\char`\ }} % (visible) space in a string
\chardef\TL=`\~ % tilde in a string
\chardef\UL=`\_ % underline character in a string

\newbox\bak \setbox\bak=\hbox to -\em{} % backspace one em
\newbox\bakk\setbox\bakk=\hbox to -2\em{} % backspace two ems

\newcount\ind % current indentation in ems
\def\1{\global\advance\ind by1\hangindent\ind\em} % indent one more notch
\def\2{\global\advance\ind by-1} % indent one less notch
\def\3#1{\hfil\penalty#10\hfilneg} % optional break within a statement
\def\4{\copy\bak} % backspace one notch
\def\5{\hfil\penalty-1\hfilneg\kern2.5\em\copy\bakk\ignorespaces}%optional break
\def\6{\ifmmode\else\par % forced break
  \hangindent\ind\em\noindent\kern\ind\em\copy\bakk\ignorespaces\fi}
\def\7{\Y\6} % forced break and a little extra space

\let\yskip=\smallskip
\def\to{\mathrel{.\,.}} % double dot, used only in math mode
\def\ellipsis{\kern5\em\smash{\vdots}\qquad\vbox to12pt{}\par}
\def\note#1#2.{\par\penalty5000
  \Y\noindent{\hangindent2\em\baselineskip10pt\eightrm#1 #2.\par}}
\def\startsection{\Q\noindent\strut{\bf\modno.\quad}}
\def\defin#1{\global\advance\ind by 2 \1\&{#1 }} % begin `define' or `format'
\def\A{\note{See also}} % cross-reference for multiply defined section names
\def\B{\mathopen{\.{@\{}}} % begin controlled comment
\def\C#1{\ifmmode\gdef\XX{\null$\null}\else\gdef\XX{}\fi % Pascal comments
  \XX\hfil\penalty-1\hfilneg\quad$\{\,$#1$\,\}$\XX}
\def\D{\defin{define}} % macro definition
\def\E{\cdot10^} % exponent in floating point constant
\def\F{\defin{format}} % format definition
\let\G=\ge % greater than or equal sign
\def\H#1{\hbox{\rm\char"7D\tt#1}} % hexadecimal constant
\let\I=\ne % unequal sign
\def\J{\.{@\&}} % TANGLE's join operation
\let\K=\gets % left arrow
\let\L=\le % less than or equal sign
\outer\def\M#1.{\def\modno{#1}\startsection\ignorespaces}
\outer\def\N#1. \[#2]#3.{\def\modno{#1}\flushout % begin starred section
  \gdef\rhead{PART #2:\uppercase{#3}} % define running headline
  \global\let\rrhead=\rhead
  \message{*\modno} % progress report
  \startsection{\bf\ignorespaces#3.\quad}\ignorespaces}
\outer\def\NN#1. \[#2]#3.{\def\modno{#1} % begin starred section on rh page
  \global\setbox\partialpage=\vbox{
    \vbox to 44pc{\box\partialpage\vfill}\break}
  \gdef\rrhead{PART #2:\uppercase{#3}} % define running headline
  \message{!\modno} % progress report
  \startsection{\bf\ignorespaces#3.\quad}\ignorespaces}
\def\O#1{\hbox{\rm\char'23\kern-.2em\it#1\/\kern.05em}} % octal constant
\def\P{\iftenpoint \ninepoint\fi
  \rightskip=0pt plus 100pt minus 10pt % go into Pascal mode
  \sfcode`;=3000
  \pretolerance 10000
  \hyphenpenalty 10000 \exhyphenpenalty 10000
  \global\ind=2 \1\ \unskip}
\def\Q{\tenpoint
  \rightskip=0pt % get out of Pascal mode
  \sfcode`;=1500 \pretolerance 200 \hyphenpenalty 50 \exhyphenpenalty 50 }
\let\R=\lnot % logical not
\let\S=\equiv % equivalence sign
\def\T{\mathclose{\.{@\}}}} % terminate controlled comment
\def\U{\note{This code is used in}} % cross-reference for uses of sections
\let\V=\lor % logical or
\let\W=\land % logical and
\def\X#1:#2\X{\ifmmode\gdef\XX{\null$\null}\else\gdef\XX{}\fi % section name
  \XX$\langle\,$#2{\sevenrm\kern.5em#1}$\,\rangle$\XX}
\def\Y{\par\yskip}
\let\Z=\let % now you can \send the control sequence \Z
\def\){\hbox{\.{@\$}}} % sign for string pool check sum
\def\]{\hbox{\.{@\\}}} % sign for forced line break
\def\=#1{\kern2pt\hbox{\vrule\vtop{\vbox{\hrule
        \hbox{\strut\kern2pt\.{#1}\kern2pt}}
      \hrule}\vrule}\kern2pt} % verbatim string
\let\~=\ignorespaces

\newcount\nrefs % total number of references in partial page
\newcount\baseno % smallest module number in partial page
\countdef\nprefs=199 % the number in \prefs
\countdef\nfrefs=220 % the number in \frefs
\toksdef\prefs=199 \toksdef\frefs=220
% \count and \toks registers 200--219 are also used to keep track of refs
\newcount\nn \newcount\mm \newcount\pp
\newdimen\nsize \newdimen\msize
\newskip\intermodskip \intermodskip=8pt plus 2pt minus 3pt
\newbox\partialpage
\newbox\newmod
\newtoks\newrefs
\newdimen\ninept \ninept=9pt

\def\uline{_}
\begingroup\catcode`\_=\active \globaldefs=1
\def\flushout{\ifvoid\partialpage\else
    \setbox0=\vsplit\partialpage to 44pc
    \shipout\vbox{
      \vbox to 3pc{
        \hbox to 35pc{\vbox to10pt{}\nineit\rhead\hfil\eightrm\folio}\vfill}
      \nointerlineskip\box0}
    \global\advance\pageno 1
    \let\=\makeref
    \ifsorted \let\readin=\readln \readrefs
    \else \immediate\write\reffile{!\the\pageno}\let\readin=\readrefs \fi
    \setbox0=\vbox{\eightpoint \hsize=11pc \catcode`\_=\active \let_=\_
      \rightskip=0pt plus 100pt minus 10pt
      \pretolerance 10000
      \hyphenpenalty 10000 \exhyphenpenalty 10000
      \noindent\vbox to1pt{}\par % 1pt = \topskip - \ninept
      \readin}
    \shipout\vbox to47pc{
      \vbox to 3pc{
        \hbox to 35pc{\vbox to10pt{}\eightrm\folio\hfil\nineit\rrhead\/}\vfill}
      \unvbox\partialpage
      \vfill
%     \ifdim\ht0>1pt \kern8pt\hrule \hbox{%
      \ifdim\ht0>1pt \kern11pt\hrule \hbox{%    % changed Oct 87 by DEK!
          \nsize=\ht0 \advance\nsize-\topskip
          \divide\nsize by 3 \divide\nsize by\ninept
          \multiply\nsize by\ninept \advance\nsize\topskip
          \vsplit0 to\nsize \kern1pc
          \msize=\ht0 \advance\msize-\topskip
          \divide\msize by 2 \divide\msize by\ninept
          \multiply\msize by\ninept \advance\msize\topskip
          \vbox to\nsize{\vsplit0 to\msize\vss}\kern1pc
          \vbox to\nsize{\box0\vss}} \fi}
    \global\advance\pageno 1 \global\let\rhead=\rrhead\fi
  {\globaldefs=1
    \nn=199 \loop \ifnum\nn<221 \toks\nn={} \count\nn=0 \advance\nn 1 \repeat
    \nrefs=0
    \baseno=\modno}}

\def\{\begingroup \catcode`\_=\active \let_=\uline \globaldefs=1
  \newrefs=\bgroup}
\def\FI{\nn=\modno \advance\nn-\baseno \advance\nn 200
  \advance\nrefs-\count\nn \nn=\nrefs
  \let\=\countnewref \the\newrefs \endgroup
  \nobreak\null\nobreak\vskip-\baselineskip\penalty-500\vfil\eject}
\def\twillout{\setbox\newmod=\vbox{\unvbox255}
  \ifnum\nn=0 \nsize=0pt
% \else \nsize=3pt \multiply\nsize\nn \advance\nsize 1pc \fi
  \else \nsize=3pt \multiply\nsize\nn \advance\nsize 1.25pc \fi % changed Oct 87!
  \ifvoid\partialpage\else\advance\nsize 5pt \advance\nsize\ht\partialpage\fi
  \advance\nsize\ht\newmod 
  \message{\the\nsize} % show the page break decision amount
  \ifdim\nsize>88pc \flushout \message{\modno}\fi
  \addtopartialpage}
\def\addtopartialpage{\let\=\addnewref \catcode`\_=\active \let_=\uline
  \globaldefs=1 \the\newrefs
  \nsize=\dp\partialpage
  \setbox\partialpage=\ifvoid\partialpage \box\newmod
  \else \vbox{\unvbox\partialpage \vfilneg
      \vskip\intermodskip \unvbox\newmod}\fi}
\endgroup
\def\shortpage{\par\vskip-2\baselineskip
  \null\vfill\penalty-5000\message{(shortpage)}}

\def\countnewref #1 #2 {\ifnum#2<\baseno \let\next\cn
  \else\ifnum#2>\modno \let\next\cn \else\let\next\flushref\fi\fi\next#1 #2 }
\def\cn #1 #2 {\expandafter\cnr\csname#1 #2\endcsname #2 }
\def\cnr#1#2 #3 {\ifx#1\relax \advance\nn 1 \fi}
\def\flushref#1 #2 #3 {}
\def\addnewref #1 #2 {\ifnum#2<\baseno \let\next\an \nn=199
  \else\ifnum#2>\modno \let\next\an \nn=#2 \advance\nn-\baseno
    \advance\nn200 \ifnum\nn>219 \nn=220 \fi
    \else\let\next\flushref\fi\fi\next#1 #2 }
\def\an #1 #2 {\expandafter\anr\csname#1 #2\endcsname #1 #2 }
\def\anr#1#2 #3 #4 {\ifx#1\relax
    \advance\nrefs1 \advance\count\nn 1
    \def#1{\#2 #3 #4 }\toks\nn=\expandafter{\the\toks\nn\#1}\fi}
\ifsorted \def\#1{\global\let#1\relax}
\else \def\#1{#1\global\let#1\relax} \fi
\def\makeref #1 #2 #3#4 {\nn=#2 \hangindent=1em
  \ifsorted\else\toks0={#1 #2 #3#4}\immediate\write\reffile{+ \the\toks0}\fi
  \noindent\\{#1}\ifx#3\\else
  \if#3:: \else\unhcopy\eqbox \fi#4\fi, \section\number\nn.\par}
\newbox\eqbox \setbox\eqbox=\hbox{\kern2pt\eightrm=\kern2pt}
\def\section{\mathhexbox278}
\def\{\}
\def\readrefs{\the\prefs \mm=\modno \advance\mm-\baseno \advance\mm200
  {\def\##1{\global\let##1\relax}\pp=200 % this patch made AFTER first edition
    \loop\ifnum\pp<\mm \the\toks\pp \advance\pp 1 \repeat}	% of Vol B!!
  \loop\ifnum\mm<221 \the\toks\mm \advance\mm 1 \repeat}
\def\readln{\read\sreffile to\next \expandafter\next\space \readin}
\def\donewithpage#1 {%\ifnum\pageno=#1 \else\message{page #1??}\fi
  \let\readin=\relax}
\def\rhead{BLANK PAGE TO THROW AWAY}

\hsize=35pc \vsize=\maxdimen \output={\twillout}

% the remaining stuff applies only to the rear matter

\newif\ifpagesaved
\def\lheader{\vbox to10pt{}\nineit\rhead\hfil\eightrm\folio}
\def\rheader{\vbox to10pt{}\eightrm\folio\hfil\nineit\rhead\/}
\def\page{\box255 }
\def\normaloutput#1{\shipout\vbox{
      \vbox to 3pc{
        \hbox to 35pc{\ifodd\pageno\rheader\else\lheader\fi}\vfill}
    \nointerlineskip#1}
  \global\advance\pageno 1 }

\newbox\sbox % saved box preceding the index
\newbox\lbox % lefthand column in the index
\def\inx{\box\partialpage
  \vskip24pt plus 1fil % we are beginning the index
  \output{\ifpagesaved\normaloutput{\box\sbox}\fi
    \global\setbox\sbox=\page \global\pagesavedtrue}
  \pagesavedfalse \eject % eject the page-so-far and predecessors
  \setbox\sbox\vbox{\unvbox\sbox} % take it out of its box
  \vsize=44pc \advance\vsize by -\ht\sbox % the remaining height
  \hsize=17pc % column width for the index (1pc between cols)
  \eightpoint
  \parfillskip 0pt plus .6\hsize % try to avoid almost empty lines
  \def\lr{L} % this tells whether the left or right column is next
  \output{\if L\lr\global\setbox\lbox=\page \gdef\lr{R}
    \else\normaloutput{\vbox to44pc{\box\sbox\vss
        \hbox to35pc{\box\lbox\hfil\page}}}
    \global\vsize=44pc\gdef\lr{L}\global\pagesavedfalse\fi}
  \message{Index:}
  \parskip 0pt plus .5pt
  \gdef\rhead{INDEX}
  \outer\def\:##1, {\par\hangindent2em\noindent##1:\kern1em} % index entry
  \let\ttentry=\. \def\.##1{\ttentry{##1\kern.2em}} % give \tt a little room
  \def\[##1]{$\underline{##1}$} % underlined index item
  \rm \rightskip0pt plus 2.5em \tolerance 10000
  \hyphenpenalty 10000 \parindent0pt}
\def\fin{\par\vfill\eject % this is done when we are ending the index
  \ifpagesaved\null\vfill\eject\fi % output a null index column
  \if L\lr\else\null\vfill\eject\fi % finish the current page
  \parfillskip 0pt plus 1fil
  \gdef\rhead{NAMES OF THE SECTIONS}
  \message{Section names:}
  \output{\normaloutput\page}
  \ninepoint
  \hsize=35pc \vsize=44pc
  \def\note##1##2.{\hfil\penalty-1\hfilneg\quad{\eightrm##1 ##2.}}
  \linepenalty=10 % try to conserve lines
  \def\U{\note{Used in}} % cross-reference for uses of sections
  \def\:{\par\hangindent 2em}\let\*=*\let\.=\ttentry}
\def\con{\par\vfill\eject\end} % finish the section names

% To produce only a subset of pages, put the page numbers on separate
% lines in a file called pages.tex
\let\Shipout=\shipout
\newread\pages \newcount\nextpage \openin\pages=pages
\def\getnextpage{\ifeof\pages\else
 {\endlinechar=-1\read\pages to\next
  \ifx\next\empty % in this case we should have eof now
  \else\global\nextpage=\next\fi}\fi}
\ifeof\pages\else\message{OK, I'll ship only the requested pages!}
 \getnextpage\fi
\def\shipout{\ifeof\pages\let\next=\Shipout
 \else\ifnum\pageno=\nextpage\getnextpage\let\next=\Shipout
  \else\let\next=\Tosspage\fi\fi \next}
\newbox\garbage \def\Tosspage{\deadcycles=0\setbox\garbage=}
